Instruction Register: nactena instrukce, Program Counter: adresa soucasne
instrukce. Existuje registr na adresu do pameti (Memory Access Register),
buffer pro pamet (Memory Buffer Register).

Von Neumann: ALU, pamet, input, output, controller.

Control-flow: standardni von Neumannova architektura, sekvencni presne podle
programu. Data-flow: instrukce se provadi jakmile jsou k nim pripravena data.

Harvardska architektura: oddelena pamet na program, klidne s jinymi parametry.

Preruseni: vnejsi, vnitrni.

Faze zpracovani instrukce: read (precteni instrukce), dekodovani, load
(operandu), execute, store (ulozeni vysledku). Posun PC.

Load-store architektura: vsechno se musi nejdriv nahrat do registru, nemaji
instrukce pro pristup rovnou do pameti.

Zapnuti PC: 1) CPU zacne vykonavat kod BIOSu (ktery posloucha na dane adrese),
2) BIOS inicializuje HW, nacte boot sektor, 3) boot sektor pomoci BIOSu nacte
a spusti zavadec OS, 4) zavadec OS pusti OS, 5) OS.

MBR = 512 bytu.

CISC a RISC. CPUcka maji FPU*, ALU*, registry.

Instrukce HLT na spani, kdyz neni potreba delat nic jineho.

Ortogonalni instrukcni sada: nepredpoklada implicitne zadne registry.

Deleni CPU: dle delky operandu (4,16,32,64,...).
Mensi sirky: embedded zarizeni.
MIPS: million instructions per second.

Pipeliny: paralelizace nezavislych kusu zpracovani instrukce.
Superskalarita: pridame nezavisle exekucni jednotky (treba vic FPU, ALU, ...).
Out of order: rozdrobeni na mikrooperace, vlozeni do vyckavaciho bufferu,
tam ceka nez dostane operandy, pak se vykona, pak se vysledky uchovaji v reorder
bufferu.

Symetrie: asymetricka je treba CPU+GPU.
SIMD: napriklad na GPUckach.

tightly-coupled, loosely-coupled

I/O zarizeni: blokove/znakove. sekvencni pristup/nahodny. komunikace:
synchronni/asynchronni. Existuji sdilena zarizeni (tiskarny se spooingem, ...).
Propojovani: dvoubodove/vicebodove (sbernice).

Port-mapped I/O: sprcialni port a instrukce na CPU.
Memory-mapped I/O: vlozeni v adresnim prostoru.

Sbernice: synchronni/asynchronni. Slave/target.

Polling, interrupt-driven I/O; programmed I/O vs. DMA

DMA: CPU dostane interrupt az skoncime, u multiprocesoru i prenos mezi jadry.
DMA radic obsahuje registry, ktere nastavi CPU.
DMA radic se sam chova jako periferie pripojena na sbernici: komunikuje s
pameti, zajistuje aby procesor zrovna sbernici nepotreboval.
Muze dojit k pametove inkoherenci, pak se to musi resit (MOESI?).

Architektury OS: monoliticka (Unix, Windows); mikrojadra.
Virtualni stroje, dnes se pouziva napriklad pro managed jazyky (.NET, Java).
OS se vola z aplikace pres preruseni.

TODO: Vztah OS a HW, obsluha preruseni.

Systemove volani: API mezi OS a userspacem.
Thready: user-level, kernel threads.

Planovani: chceme spravedlive, efektnivni, rychle.
Kriteria procesu: vazanost na CPU nebo I/O. Davkovy/interaktivni? Priorita?

Algoritmy: first come first served; round robin; vice front (z i-te fronty
kdyz jsou prvni i fronty prazdne). I/O => posun do i-1, preempce => i+1.

Busy waiting.

TODO: dusledne stridani (promenna 'turn')

TODO: podivat se na Petersonovo reseni?

Semafor: zámek s větším N, hodi se na producent/konzument. Pamatuje si
pocet slotu; zabrani/uvolneni.
Mutex: semafor pro 0/1.
Monitor: mutex + condition variable.

Coffmanovy podminky pro deadlock: vylucny pristup, drz a cekej (procesy
muzou zadat o dalsi prostredky), neodnimatelnost, cekani do kruhu.

Deadlocky: detekce a zotaveni: waits-for graf; priority vlaken (jedno vyhraje);
zabijeni cyklu. Predchazeni: spooling (sprava prostredku); zadej o vsechno hned;
odnimani prostredku; cekani do kruhu: zadej jenom ve vzestupnem poradi (dle
indexu prostredku).

Pres prioritu: wound-wait (zabij min dulezite vlakno kdyz ma muj zamek);
wait-die (umri kdyz bych si chtel vzit zamek od dulezitejsiho vlakna).

Dvoufazove zamykani: nejdriv zamykej, pak jenom odmykej.

Pametova hierarchie, cache.

Preklad adresy, virtualni pamet.

Fungovani mallocu: bloky nejakych velikosti.
First-fit: prvni volny dostatecne velikosti; obcas rozdeli diru.
Next-fit: dalsi volny dostatecne velikost, zacni od posledni prohledane pozice.
Best-fit: nejmensi volny dostatecne velikosti; pomaly (cely seznam), nechava
velke diry.
Buddy system: bloky velikosti $2^n$, slucuj volne vedle sebe.

Fragmentace: externi (nepouzite bloky); interni (nevyuzije cely prideleny
prostor vevnitr bloku); sesypani nejde.

MMU: prevod mezi virtualni a fyzickou pameti.
Po strankach. Page Table Base Register.
Translation Lookaside Buffer.

Page Table Base Register: ukazatel na tabulku prekladu stranek.
Algoritmy na vyhazovani: OPT, FIFO, LRU napriklad.

Soubory: alokace: souvisla, spojovy seznam, indexova (UNIX inode s direct,
indirect, double indirect, triple indirect).
Symbolic/hard linky.
Hierarchie adresaru (strom/DAG).
Implementace adresaru: pevne velke zaznamy, spojak, B-stromy.

Velikost MBR = 512 B.

RAID: JBOS (just bunch of disks).
0 = striping bez redundance. 1 = mirroring, 2 = 7-bitovy Hamming kod,
3 = 1 paritni disk, po bitech na disky, 4 = 1 paritni disk a striping,
5 = distribuovana parita a striping, 6 = ???
