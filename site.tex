\documentclass[a4paper,10pt,titlepage]{article} \usepackage[utf8]{inputenc}
\usepackage{a4wide} \usepackage[czech]{babel}
\usepackage[small,compact]{titlesec}

\usepackage{graphicx}
\usepackage{amsthm}
\usepackage{amsmath}
\usepackage{amsfonts}
\usepackage{amssymb}

\newtheorem{theorem}{Věta}
\newtheorem{define}{Definice}
\newtheorem*{notation}{Značení}
\newtheorem*{example}{Příklad}
\newtheorem*{remark}{Poznámka}

\begin{document} \pagestyle{empty}

LAN = local, WAN = wide area network. MAN = metropolitan area network.
PAN = personal area network, napriklad Bluetooth

LAN: sdili se prostredky - tiskarny, servery, ...
Byvala vyssi rychlost.
LAN jsou privatni.

WAN: vzdaleny pristup, distribuovane.
WAN jsou verejne.

Sitove modely: ISO/OSI. Sitova architektura je konkretni implementace sitoveho
modelu.

OSI model (open systems interconnection): 7 vrstev:
\begin{enumerate}
\item physical: jak proudí bity po drátech
\item link: prenos dat mezi sousedy na jedne siti
\item network: dosazeni cile, i skrz hranice siti. Tady se treba routuje.
\item transport: TCP/UDP. "end-to-end přenos datových celků"
\item session
\item presentation: RPC protokoly?
\item application: komunikace mezi programy
\end{enumerate}

TCP/IP model odpovídá plus minus OSI:
\begin{enumerate}
\item síťové rozhraní = OSI 1+2. Ethernet, WiFi, PPP, ...
\item síťová vrstva = OSI 3. IP/ARP
\item transportní = OSI 4. TCP/UDP/ICMP/...
\item aplikační = OSI 5+. FTP/HTTP, DNS, NFS/XDR/RPC, ...
\end{enumerate}

Služby se adresují URI (Uniform Resource Identifier):
schéma://jméno:heslo@adresa:port[cesta]

Adresy: MAC na linkové vrstvě, IP na link vrstvě, doména na aplikační vrstvě

ICANN: spravuje domény nejvyšší úrovně.
CZ.NIC spravuje doménu .cz.

OS se podívá na PDU na link vrstvě, zjistí protokol network vrstvy a pošle
příslušnému stacku (IP stack, IPX, ARP, ...).
IP stack se podívá na typ transportního protokolu (TCP/UDP) a pošle dál,
TCP/UDP se podívají na port a pošlou to aplikaci.

Porty jsou 16bitové. Porty < 1024 jsou rezervované. Např. 80 = HTTP.
Když se baví klient a server, server má většinou nějaký známý port.
Klient pošle serveru packet s např. destination portem 80 jako HTTP,
a vymyslí si nějaký source port, na kterém bude čekat na odpověď od serveru.

Porty:
\begin{itemize}
\item 21: FTP (TODO: plus nějaký další port... FTP používá dva)
\item 22: SSH
\item (23: Telnet)
\item 80: HTTP, 443: HTTPS
\item (53 UDP/TCP: DNS), (67,68/UDP: DHCP)
\end{itemize}

Socket je jeden konec komunikačního kanálu: IP + port.

DNS: má primární, sekundární a caching servery. Primární je autorita,
sekundární má kopie dat, caching-only jenom cachuje.
Záznamy: NS/SOA: nameservery a info o doméně, A/AAAA: normální záznam IP4/IP6.
PTR: reverzní záznamy (1.1.168.192.in-addr.arpa ==> www.seznam.cz). CNAME.
MX: záznamy pro mailservery.

DNS záznamy mají TTL, po jaký se můžou udržovat v cachích.

Rekurzivni, nerekurzivni odpovedi: nerekurzivni se jenom podiva do svych
zaznamu, rekurzivni se muze zeptat jinde.
DNS diagnostika přes nslookup, dig.

FTP protokol: velmi starý. anonymous:(mail).
Kódy odpovědí: 1xx = předběžně kladná odpověď, 2xx = definitivní kladná,
3xx = prozatímní (potřebuju další příkazy), 4xx = dočasná záporná, 5xx = trvalá
záporná. Podobné schema mají další protokoly.

TODO: aktivní a pasivní mód, nechápeme. DOPLNIT.

\section{Elektronická pošta}
Mail user agent, mail transfer agent.
MUA: běží u uživatele. MTA: SMTP se používá na posílání, POP/IMAP na přijímání.

SMTP protokol: příkazy: "HELO (kdo jsem ja)", "MAIL FROM: <...>", "RCPT TO:
<...>", "DATA", dopis, konci se teckou na samostatnem radku.
Hlavicky uvnitr dopisu jsou nezavisle.

Multipart encoding (multipart/mixed): balení víc souborů v mailu, s oddělovači.
MIME typy: text/plain, text/html, ...

Hlavičky: Date, From = autor, Sender = kdo to poslal, Reply-To, To = adresáti,
Cc = carbon copy, Bcc = blind carbon copy, Message-ID, Subject, Received =
záznam o průběhu doručení

Base64: překódování, Quoted-Printable: non-ASCII jako "=HexHex"

POP, IMAP na vyzvedávání pošty. IMAP je novější a lepší, má zabudované
šifrování.

\section{HTTP}
HTTP: WWW stránky v HTML.

GET (cesta) HTTP/1.1
Host: www.cuni.cz

==>

HTTP/1.1 200 OK
Content-Type: ...
(... ...)

Chybové kódy:
1xx: informativní (odpověď přijata, zpracovává se)
	TODO: jak se to v praxi použije?
2xx: kladná definitivní
3xx: prozatímní kladná, nutné poslat další požadavky. Například redirecty.
4xx: chyba na straně klienta
5xx: chyba na straně serveru

GET (jde cachovat), HEAD (jen hlavičky), POST, PUT (mají nějaké tělo požadavku),
CONNECT (na tunelování)

\section{Routing}
TCP otevreni: SYN - SYN+ACK - ACK, uzavreni: FIN+ACK - ACK

TCP paket: source, dest. port, sequence number, ACK number, data offset, flagy,
velikost okna, checksum, ...

UDP paket: source+dest. port, velikost, checksum -- protoze to je proudovy
protokol.

Třídy IP adres jsou A, B, C, D, E. Třída IP se určí podle prvních bitů prvního
oktetu.
\begin{itemize}
\item 0 => Ačková adresa, za ní 7 bitů co dávají síť.
\item 10 => B, za ní 14 bitů na síť.
\item 110 => C, za ní 21 bitů na síť.
\item 1110 => D: multicast.
\item 1111 => E: experimental.
\end{itemize}

Speciální adresy:
\begin{itemize}
\item 127.* = loopbacky
\item 10.*/8, 172.16-31/16, 192.168.*.0/24 = privátní sítě
\item network broadcast: (síť) (jedničky)
\item broadcast všem v síti: 255.255.255.255
\end{itemize}

Subnetting: dneska se vpodstatě třídy adres ignorují, používají se subnety,
adresa sítě je nějaký prefix IP. V jedné síti různé masky => VLSM.

IP datagram: protokol, TTL, source/destination IP.
TODO: fragmentace...?

IPv6: 128 bitu na adresu.

Směrovací tabulky: od specifických po obecné záznamy: cíl, maska, gateway.
Pakety se posílají s nezměněnými IP adresami.
Záznamy: staticky ručně vyrobené, nebo dynamicky vytvořené.
Dynamické: např. ICMP Redirect, RIP, OSPF, BGP, atd.

Distance vector algoritmy: záznamy si pamatují i "vzdálenost".
Například RIP: metrika = počet hopů. Bellman-Fordův algoritmus.

Existuje filtrování: na IP, na HTTP, atd (zakážu komunikovat po nějakém portu).

Překlad adres (slide 135).

ARP: konverze MAC a IP. Každý stroj má ARP cache.

\end{document}
