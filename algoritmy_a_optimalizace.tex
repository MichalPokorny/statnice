\documentclass[a4paper,10pt,titlepage]{article} \usepackage[utf8]{inputenc}
\usepackage{a4wide} \usepackage[czech]{babel}
\usepackage[small,compact]{titlesec}

\usepackage{graphicx}
\usepackage{amsthm}
\usepackage{amsmath}
\usepackage{amsfonts}
\usepackage{amssymb}

\def\R{\mathbb{R}}
\def\Z{\mathbb{Z}}

\begin{document} \pagestyle{empty}
\chapter{Algoritmy a optimalizace}

\begin{itemize}
\item Aproximační algoritmy pro SAT, nezávislé množiny, množinové pokrytí,
	rozvrhování.
\item Použití lineárního programování pro aproximační algoritmy.
\item Využití pravděpodobnosti při návrhu algoritmů.
\item Voroného diagramy.
\item Aranžmá nadrovin.
\item Incidence bodů a přímek.
\item Základní algoritmy výpočetní geometrie.
\item Samoopravné kódy.
\item Pravděpodobnostní metoda - příklady použití.
\end{itemize}

Vycházím ze svých zápisků z Aproximačních a pravděpodobnostních algoritmů.
Asi bude potřeba, abych se taky podíval do Kombinatorické a výpočetní geometrie.
Počítám, že samoopravné kódy a pravděpodobnostní metodu ukradnu od Verči.

\section{Aproximační algoritmy pro SAT, nezávislé množiny, množinové pokrytí,
	rozvrhování}

\subsection{SAT}
Aproximační verze SATu je Max-SAT: snažíme se splnit co nejvíce klauzulí umíme.
Pro pořádek předpokládáme, že klauzule jsou neprázdné a nejsou tautologie
(tj. neobsahují $x\vee\neg x$).
Max-$k$-SAT: délka klauzule nejvíc $k$, E$k$-SAT: přesně $k$.

Max-2-SAT je NP-těžký.

\begin{itemize}
\item Rand-SAT: Náhodně nezávislý výběr hodnoty každé proměnné: každá klauzule
	platí s pravděpodobností $1-2^{-k}$, pro Max-E3-SAT dostaneme
	7/8-aproximační algoritmus. Pro Max-E3-SAT nic lepšího nejde.

\item Biased-SAT: podíváme se pro každou proměnnou, jestli se vícekrát vyskytne
	jako jednotková klauzule $x$, nebo $\neg x$. Když se $x$ vyskytuje
	víckrát než $\neg x$, zvol $x=1$ s pravděpodobností
	$\phi-1=(\sqrt{5}-1)/2\approx 0.618$, jinak obráceně ($2-\phi$).

	Biased-SAT je $\phi-1$-aproximační.
	Bez újmy na obecnosti vyruš po dvojicích jednotkové klauzule $(x_i)$
	a $(\neg x_i)$ -- platí jenom jedna polovina z nich, tohle vyrušení
	nezhorší aproximační poměr.
	TODO: Tohle Prvák nechápe.

	Klauzule délky 1 je splněna s pravděpodobností $\phi-1$,
	ostatní splněny s pravděpodobností aspoň $1-(\phi-1)^k\geq
	1-(\phi-1)^2=\phi-1$.

\item LP-SAT: Použijeme lineární program:
	definuj pro klauzuli $C_t$ funkci:
	$f(C_t)=\sum_{(x_i)\in C_t} y_i + \sum_{(\neg x_i)\in C_t} (1-y_i)$.
	Maximalizuj $\sum z_j$ pro $y_i\in [0;1], z_i\in [0;1],
	(\forall t) z_t\leq f(C_t)$.
	Proměnné jsou tedy vlastně "relaxovaná pravdivost proměnných" a
	"pravdivost klauzulí".

	Vyřeš lineární program a pak vyber pravdivosti nezávisle náhodně
	podle $y_i^*$.
	LP-SAT je $(1-\frac{1}{e})$-aproximační. ($\approx 0.632$)

	Lemma: $P[C_j(\overrightarrow{a})=1]\geq (1-(1-\frac{1}{k})^k)z_j^*$.

	Důkaz: Bez újmy na obecnosti nechť je klauzule $C_j$ složena
	z $k$ pozitivních literálů. $P[C_j(\overrightarrow{a})=0]=
	\prod(1-y_i)^*\leq
	(\sum{\frac{1-y_i^*}{k}})^k=(1-\frac{f(C_j)}{k})^k\leq
	(1-\frac{z_j^*}{k})^k$.

	Odhadneme tedy $P[C_j(\overrightarrow{a})=1]\geq
	1-(1-\frac{z_j^*}{k})^k$. $g(i)=1-(1-\frac{i}{k})^k$ je konvexní,
	tak vemu $g(z_j^*)\geq (g(1)-g(0))\cdot z_j^*$, dostanu
	$P[C_j(\overrightarrow{a}=1)]\geq (1-(1-\frac{1}{k})^k)z_j^*$.

	Z toho vyjde, že to je $1-\frac{1}{e}$-aproximační.

\item Best-SAT: s pravděpodobností $\frac{1}{2}$ vezmu Rand-SAT, jinak LP-SAT.
	Rozeberu si případy délek klauzulí 1, 2, 3+.
	Dostanu aproximační poměr $3/4$.
\end{itemize}

\subsection{Nezávislé množiny}
Pro $\varepsilon>0$ nejdou maximální nezávislá množina/maximální klika
aproximovat s poměrem $\O(n^{\varepsilon-1})$ pokud $P\neq NP$.

%%%%Hledáme největší nezávislou množinu v grafu.
%%%%Aproximace s konstantním poměrem se nedá v polynomiálním čase, pokud $P\neq NP$.
%%%%
%%%%Algoritmus: $I\gets\emptyset$, $d_v$ buď stupeň $v$. Dokud $V\neq\emptyset$:
%%%%\begin{enumerate}
%%%%\item Paralelně pro každý vrchol: když $d_v=0$, tak $I+=\{v\}$, $V-=\{v\}$.
%%%%\item Paralelně pro každý vrchol: označ jej s pravděpodobností $\frac{1}{2d_v}$.
%%%%\item Paralelně pro každou hranu $\{u,v\}$: když jsou $\{u,v\}$ oba označené,
%%%%	odeber značku z toho z nich, co má menší stupeň.
%%%%\item Ať $S$ jsou označené vrcholy a $N(S)$ jejich sousedé. $I+=S$ a $V-=(S\cup
%%%%	N(S))$ a odeber hrany incidentní k $S\cup N(S)$.
%%%%\end{enumerate}
%%%%
%%%%Vrchol je \emph{dobrý}, pokud má aspoň $d_v/3$ sousedů stupně nejvýše $d_v$,
%%%%ostatní jsou \emph{špatné}. Hrana je dobrá, obsahuje-li dobrý vrchol, jinak je
%%%%špatná.
%%%%
%%%%Lemma 1: je-li vrchol $v$ dobrý, tak nějakého jeho souseda označím s
%%%%pravděpodobností $\geq 0.15$.
%%%%Důkaz: $P[\text{neoznačím žádného souseda }v]\leq (1-\frac{1}{2d_v})^{d_v/3}\leq
%%%%	e^{-1/6}<0.85$
%%%%
%%%%Lemma 2: Je-li vrchol $v$ označený, pak se dostane do $S$ s pravděpodobností
%%%%	$\geq 1/2$.
%%%%Důkaz: $P[\text{odeberu mu značku}]\leq d(v)\cdot\frac{1}{2d(v)}=1/2$
%%%%
%%%%Lemma 3: Dobrých hran je aspoň půlka.
%%%%Důkaz: Zorientuj $G$: hrany z malých stupňů do velkých, vždy jenom jedna.
%%%%	Je-li $v$ špatný, tak do něj vede nejvýš $d_v/3$ hran.
%%%%	$E(V\imp v)\leq\frac{1}{2}E(v\imp V)$,
%%%%	takže $2E(B\imp v)\leq E(v\imp V)=E(v\imp B)+E(v\imp G)$.
%%%%	Posčítám to pro všechny špatné vrcholy, dostanu:
%%%%	$2E(B\imp B)\leq E(B\imp B)+E(B\imp G)$.
%%%%
%%%%Po $\O(\log N)$ krocích nám zmizí všechny vrcholy.

TODO: A jaky to ma aproximacni pomer? A proc?
TODO: analyza

\subsection{Množinové pokrytí}
Máme systém všech prvků $\{1, \ldots, n\}$ a systém jeho podmnožin $S_1, \ldots,
S_m$. Každá podmnožina stojí $c_i$. Pro pozdější analýzu zavádíme $g=max|S_i|=$
``největší množina''; $f=$``v kolika množinách je nejčastější prvek''. Cílem je
nakoupit takové množiny, že budeme mít všechny prvky systému a celkem zaplatíme
minimum.

\begin{itemize}
\item	Hladový algoritmus v každém kroku uvažuje pouze $S_j$ takové, že je má smysl koupit.
	Koupí vždy tu, který má nejlepší poměr ``cena/výkon'', tedy v každém kroku kupuji
	množinu minimalizující $\frac{c_j}{S_j \setminus E}$, kde $E$ je množina již koupených
	prvků. Algoritmus dává ne moc dobrý $H_g$-apx poměr (harmonické číslo $g$).
	Analýza je jednoduchá přes zjevný LP program a jeho duál (+komplementarita).

\item	Primárně-duální algoritmus řeší duální problém. V každém kroku kupuju množiny takové,
	že splňují duální LP s rovností (tedy pro nepokrytý prvek jsou duální podmínky
	ostré). Algoritmus je $f$-apx; snadnou analýzou LP a duálu.

	$$
	ALG = \sum\limits_{koupene} c_j = \sum\limits_{koupene}\sum\limits_{element} y_e \leq f \sum\limits_{1}^{n} y_e \leq f.OPT
	$$
\end{itemize}

\subsection{Rozvrhování}
Máme $p_j$ úloh, které rozvrhujeme na jeden z $m$ identických strojů.
Úloha běží v čase $[S_j, C_j=S_j+p_j)$. Minimalizujeme délku rozvrhu.
Je to NP-kompletní (například přes subset-sum).

\begin{itemize}
\item Hladový algoritmus: úlohu polož na nejméně zatížený stroj.
	Říká se mu taky "list scheduling".
	Věta: hladový algoritmus je $(2-1/m)$-aproximační.

	Důkaz:
	Bez újmy na obecnosti: poslední dokončená úloha je poslední rozvržená (rozvrh
	by se nezkrátil, kdybych zapomněl na všechny za ní v seznamu úloh).
	$C_n-S_n=p_n\leq OPT$, protože optimální rozvrh rozvrhne mimo jiné $p_n$.
	Do času $S_n$ všechny stroje běží naplno, takže $S_n\leq OPT$.

	$$S_n\leq \sum_{j=1}^n \frac{p_j}{m} - \frac{p_n}{m}\leq OPT-\frac{p_n}{m}$$
	Z tohohle vytáhnu $C_n\leq OPT+(1-\frac{1}{m})p_n\leq (2-\frac{1}{m})OPT$.
\item LPT algoritmus: $(4/3-1/3m)$-aproximační.
	Nejdřív setříď joby sestupně podle délky, pak hladově pokládej na
	nejméně zatížený stroj.

	Když $p_n\leq OPT/3$, tak $S_n\leq (\sum p_j/m)-p_n/m\leq OPT-p_n/m$,
	tak $C_n\leq OPT+(1-\frac{1}{m})p_n\leq (4/3-1/3m)OPT$.
	Když naopak $p_n>OPT/3$, tak $OPT$ má na každém stroji nejvýš
	2 úlohy.

	Když $n\geq m+1$, tak $p_m+p_{m+1}\leq OPT$ (tyhle dvě jsou nejmenší).
	Když $n\geq m+2$, tak z prvních $m+2$ úloh jsou dvě dvojice na jednom
	stroji, aspoň jedna úloha aspoň velká $p_{m-1}$ je ve dvojici,
	tak to bude dvojice $p_{m-1}+p_{m+2}$. Tohle přesně algoritmus
	vyrábí, tak dává optimální rozvrh.
\end{itemize}

\section{Použití lineárního programování pro aproximační algoritmy}
Viz LP-SAT.

\section{Využití pravděpodobnosti při návrhu algoritmů}
TODO

Viz QuickSort, Best-SAT, atd.

\section{Voroného diagramy}
Voroného diagram pro body $B$ je rozdělení na buňky podle nejbližšího
bodu. Uvažujeme $\R^2$ s Euklidovskou metrikou, i když to jde zadefinovat
i v jiných MP.
Voroného diagram je duální k Delaunayově triangulaci.
Buňky jsou konvexní mnohostěny, protože vzniknou průnikem poloprostorů.
Dva body sousedí na konvexním obalu právě když jejich buňky sdílejí nekonečnou
hranu.

Voroného diagramy jdou vygenerovat Fortunovým algoritmem, který potřebuje
prostor $\O(N)$ a čas $\O(N\log N)$. Zametáme zleva
doprava, máme dvě zametací čáry: "beach line" a "sweep line".
Sweep line je rovná čára, beach line se skládá z kousků parabol.
Beach line dělí rovinu na část, kde triangulaci známe, a část, kde ji neznáme.

Každý bod nalevo od sweep line společně se sweep line indikuje parabolu,
která ho odděluje od bodů bližších ke sweep line.
Držím si \emph{binární strom} popisující beach line a \emph{prioritní frontu}
událostí, které můžou změnit strukturu beach line.
Eventy jsou:
\begin{itemize}
\item Přidání paraboly, když sweep line projde novým bodem.
\item Odebrání paraboly. Ať jsou za sebou paraboly dané vrcholy $a,b,c$.
Nastane, jakmile se paraboly $a,c$ protnou v jednom bodě s parabolu $b$.
\end{itemize}

TODO: Víc detailů?

\section{Aranžmá nadrovin}
TODO

\section{Incidence bodů a přímek}
Pro $G$ zakreslený do roviny v obecné poloze platí
\begin{itemize}
\item $m \geq 4n \quad \Rightarrow \quad \#pruseciku \geq \frac{1}{64}\frac{m^3}{n^2}$.
\quad Důkaz pstní metodou - vybíráme náhodný podgraf, každý vrchol s pstí $\frac{4n}{m}$.
\end{itemize}

\noindent Pro $l$ přímek a $n$ bodů v rovině platí
\begin{itemize}
\item $\#incidenci \leq 4((n.l)^{2/3} + n + l)$
\item $2 \leq k \leq \sqrt{n} \quad \Rightarrow \quad \#(primek obsahujicich \leq k bodu) \leq O(\frac{n^2}{k^3})$
\end{itemize}

\noindent Počet jednotkových vzdáleností mezi $n$ body v rovině $\leq O(n^{3/4}).$

\section{Základní algoritmy výpočetní geometrie}
TODO

\def\vykradeno{Nestydatě vykradeno od Verči/PJK z
github.com/PJK/mff-diskretni-struktury v commitu b74b55ae4.}

\section{Samoopravné kódy}
\vykradeno

Kód na $\Z_2$, přenos nad šumivým kanélem. Navrhujeme kódování.

Trojitý opakovací kód: umíme opravit 1 chybu (narozdíl od zdvojení).

Hammingova váha, Hammingova vzdálenost, nejbližší slovo. Spolehlivost kanálu:
pravděpodobnost korektního přenosu, výskyt nezávislonáhodný.

Umět opravit $t$ chyb na slovo $\leftrightarrow$ v žádné kouli v daném slově o
poloměru $2t$ nesmí ležet jiné slovo.

Hammingův/(7,4,3) kód: 4 bity informace, 3 kontrolní bity $a_0, a_1, a_2$:
$a_0=a_3+a_4+a_6$, $a_1=a_3+a_5+a_6$, $a_2=a_4+a_5+a_6$. Z toho vyjde jeho
kontrolní matice.

Kód nad $\Z_2^n$ je $C\subseteq\Z_2^n$. Vzdálenost kódu $\Delta C:=\min_{x\neq
y\in C} d(x,y)$.

$(n,k,d)$ kód je nad $\Z_2^n$, má $|C|=2^k$ a $d=\Delta C$.
Kódování je bijekce. Dekódování přiřadí slovům kódu nejbližší kódované slovo.

Kód $C$ opravuje $t$ chyb $\leftrightarrow$ pro kazde zakodovane slovo
existuje nejvys jeden kod co je nejvys $t$ od nej.

Pozorovani: $(n,k,d)$-kod opravuje $t$-nasobne chyby prave kdyz $d\geq 2t+1$.

Generující matice lineárního $(n,k,d)$-kódu: matice $G$ tvaru $k\times n$,
radky jsou baze $C$. Zobrazeni $f(x):=x\cdot G$ je kodovani.
Pro lineární kód: $\Delta C=\min_{x\in C\smallsetminus\{0\}}\|x\|$.

Kontrolní matice lineárního $(n,k,d)$-kódu: $K\in C^{(n-k)\times k}$, řádky
jsou báze $C^\perp$. $x\in C\leftrightarrow K\cdot x^T=0$. Kód obsahuje slovo
váhy $t$ právě když $K$ obsahuje $t$ sloupců, co se sečtou na 0.
Speciálně když má různé nenulové sloupce, tak $\Delta\geq 3$.

Hammingův kód $\mathbb{H}_m$: $K$ má $m$ řádků a $2^{m-1}$ sloupců,
sloupce jsou všechny různé vektory $\Z_2^m\smallsetminus\{0\}$.

Hammingův kód je $(n,k,d)$-kód, kde $n=2^m-1$, $k=2^m-1-m$, $d=3$.

\section{Pravděpodobnostní metoda - příklady použití}
\vykradeno

Příklad použití pravděpodobnostní metody:

Množiny $S_1\ldots S_m$, kde $|S_i|=\ell$. Když $m<2^{l-1}$, tak jde sjednocení
obarvit tak, že žádná není monochromatická.
Dk: Pravděpodobnost monochromatičnosti: $1/2^l + 1/2^l = 1/(2^{l-1})$,
takže pravděpodobnost libovolné monochromatičnosti je nejvýš 1.

TODO: Víc.

\end{document}
