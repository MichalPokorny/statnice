\documentclass[a4paper,10pt,titlepage]{article} \usepackage[utf8]{inputenc}
\usepackage{a4wide} \usepackage[czech]{babel}
\usepackage[small,compact]{titlesec}

\usepackage{graphicx}
\usepackage{amsthm}
\usepackage{amsmath}
\usepackage{amsfonts}
\usepackage{amssymb}

\begin{document} \pagestyle{empty}
\chapter{Algoritmy a optimalizace}

\begin{itemize}
\item Aproximační algoritmy pro SAT, nezávislé množiny, množinové pokrytí,
	rozvrhování.
\item Použití lineárního programování pro aproximační algoritmy.
\item Využití pravděpodobnosti při návrhu algoritmů.
\item Voroného diagramy.
\item Aranžmá nadrovin.
\item Incidence bodů a přímek.
\item Základní algoritmy výpočetní geometrie.
\item Samoopravné kódy.
\item Pravděpodobnostní metoda - příklady použití.
\end{itemize}

Vycházím ze svých zápisků z Aproximačních a pravděpodobnostních algoritmů.
Asi bude potřeba, abych se taky podíval do Kombinatorické a výpočetní geometrie.
Počítám, že samoopravné kódy a pravděpodobnostní metodu ukradnu od Verči.

\section{Aproximační algoritmy pro SAT, nezávislé množiny, množinové pokrytí,
	rozvrhování}

\subsection{SAT}
Aproximační verze SATu je Max-SAT: snažíme se splnit co nejvíce klauzulí umíme.
Pro pořádek předpokládáme, že klauzule jsou neprázdné a nejsou tautologie
(tj. neobsahují $x\vee\neg x$).
Max-$k$-SAT: délka klauzule nejvíc $k$, E$k$-SAT: přesně $k$.

Max-2-SAT je NP-těžký.

\begin{itemize}
\item Rand-SAT: Náhodně nezávislý výběr hodnoty každé proměnné: každá klauzule
	platí s pravděpodobností $1-2^{-k}$, pro Max-E3-SAT dostaneme
	7/8-aproximační algoritmus. Pro Max-E3-SAT nic lepšího nejde.

\item Biased-SAT: podíváme se pro každou proměnnou, jestli se vícekrát vyskytne
	jako jednotková klauzule $x$, nebo $\neg x$. Když se $x$ vyskytuje
	víckrát než $\neg x$, zvol $x=1$ s pravděpodobností
	$\phi-1=(\sqrt{5}-1)/2\approx 0.618$, jinak obráceně ($2-\phi$).

	Biased-SAT je $\phi-1$-aproximační.
	Bez újmy na obecnosti vyruš po dvojicích jednotkové klauzule $(x_i)$
	a $(\neg x_i)$ -- platí jenom jedna polovina z nich, tohle vyrušení
	nezhorší aproximační poměr.
	TODO: Tohle Prvák nechápe.

	Klauzule délky 1 je splněna s pravděpodobností $\phi-1$,
	ostatní splněny s pravděpodobností aspoň $1-(\phi-1)^k\geq
	1-(\phi-1)^2=\phi-1$.

\item LP-SAT: Použijeme lineární program:
	definuj pro klauzuli $C_t$ funkci:
	$f(C_t)=\sum_{(x_i)\in C_t} y_i + \sum_{(\neg x_i)\in C_t} (1-y_i)$.
	Maximalizuj $\sum z_j$ pro $y_i\in [0;1], z_i\in [0;1],
	(\forall t) z_t\leq f(C_t)$.
	Proměnné jsou tedy vlastně "relaxovaná pravdivost proměnných" a
	"pravdivost klauzulí".

	Vyřeš lineární program a pak vyber pravdivosti nezávisle náhodně
	podle $y_i^*$.
	LP-SAT je $(1-\frac{1}{e})$-aproximační. ($\approx 0.632$)

	Lemma: $P[C_j(\overrightarrow{a})=1]\geq (1-(1-\frac{1}{k})^k)z_j^*$.

	Důkaz: Bez újmy na obecnosti nechť je klauzule $C_j$ složena
	z $k$ pozitivních literálů. $P[C_j(\overrightarrow{a})=0]=
	\prod(1-y_i)^*\leq
	(\sum{\frac{1-y_i^*}{k}})^k=(1-\frac{f(C_j)}{k})^k\leq
	(1-\frac{z_j^*}{k})^k$.

	Odhadneme tedy $P[C_j(\overrightarrow{a})=1]\geq
	1-(1-\frac{z_j^*}{k})^k$. $g(i)=1-(1-\frac{i}{k})^k$ je konvexní,
	tak vemu $g(z_j^*)\geq (g(1)-g(0))\cdot z_j^*$, dostanu
	$P[C_j(\overrightarrow{a}=1)]\geq (1-(1-\frac{1}{k})^k)z_j^*$.

	Z toho vyjde, že to je $1-\frac{1}{e}$-aproximační.

\item Best-SAT: s pravděpodobností $\frac{1}{2}$ vezmu Rand-SAT, jinak LP-SAT.
	Rozeberu si případy délek klauzulí 1, 2, 3+.
	Dostanu aproximační poměr $3/4$.
\end{itemize}

\subsection{Nezávislé množiny}
TODO: pro nezavisle mnoziny

\subsection{Množinové pokrytí}
TODO: pro mnozinove pokryti

\subsection{Rozvrhování}
Máme $p_j$ úloh, které rozvrhujeme na jeden z $m$ identických strojů.
Úloha běží v čase $[S_j, C_j=S_j+p_j)$. Minimalizujeme délku rozvrhu.
Je to NP-kompletní (například přes subset-sum).

\begin{itemize}
\item Hladový algoritmus: úlohu polož na nejméně zatížený stroj.
	Říká se mu taky "list scheduling".
	Věta: hladový algoritmus je $(2-1/m)$-aproximační.

	Důkaz:
	Bez újmy na obecnosti: poslední dokončená úloha je poslední rozvržená (rozvrh
	by se nezkrátil, kdybych zapomněl na všechny za ní v seznamu úloh).
	$C_n-S_n=p_n\leq OPT$, protože optimální rozvrh rozvrhne mimo jiné $p_n$.
	Do času $S_n$ všechny stroje běží naplno, takže $S_n\leq OPT$.

	$$S_n\leq \sum_{j=1}^n \frac{p_j}{m} - \frac{p_n}{m}\leq OPT-\frac{p_n}{m}$$
	Z tohohle vytáhnu $C_n\leq OPT+(1-\frac{1}{m})p_n\leq (2-\frac{1}{m})OPT$.
\item LPT algoritmus: $(4/3-1/3m)$-aproximační.
	Nejdřív setříď joby sestupně podle délky, pak hladově pokládej na
	nejméně zatížený stroj.

	Když $p_n\leq OPT/3$, tak $S_n\leq (\sum p_j/m)-p_n/m\leq OPT-p_n/m$,
	tak $C_n\leq OPT+(1-\frac{1}{m})p_n\leq (4/3-1/3m)OPT$.
	Když naopak $p_n>OPT/3$, tak $OPT$ má na každém stroji nejvýš
	2 úlohy.

	Když $n\geq m+1$, tak $p_m+p_{m+1}\leq OPT$ (tyhle dvě jsou nejmenší).
	Když $n\geq m+2$, tak z prvních $m+2$ úloh jsou dvě dvojice na jednom
	stroji, aspoň jedna úloha aspoň velká $p_{m-1}$ je ve dvojici,
	tak to bude dvojice $p_{m-1}+p_{m+2}$. Tohle přesně algoritmus
	vyrábí, tak dává optimální rozvrh.
\end{itemize}

\section{Použití lineárního programování pro aproximační algoritmy}
TODO

\section{Využití pravděpodobnosti při návrhu algoritmů}
TODO

\section{Voroného diagramy}
TODO

\section{Aranžmá nadrovin}
TODO

\section{Incidence bodů a přímek}
TODO

\section{Základní algoritmy výpočetní geometrie}
TODO

\section{Samoopravné kódy}
TODO

\section{Pravděpodobnostní metoda - příklady použití}
TODO

\end{document}
