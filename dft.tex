\documentclass[a4paper,10pt,titlepage]{article} \usepackage[utf8]{inputenc}
\usepackage{a4wide} \usepackage[czech]{babel}
\usepackage[small,compact]{titlesec}

\usepackage{graphicx}
\usepackage{amsthm}
\usepackage{amsmath}
\usepackage{amsfonts}
\usepackage{amssymb}

\newtheorem{theorem}{Věta}
\newtheorem{define}{Definice}
\newtheorem*{notation}{Značení}
\newtheorem*{example}{Příklad}
\newtheorem*{remark}{Poznámka}

\begin{document} \pagestyle{empty}
Disketni Fourierova transformace prevadi polynom reprezentovany jako koeficienty
$p_i$: $P(x)=\sum p_i x^i$ na jeho hodnoty v $\omega^i$, kde $i\in\{0,\ldots
n-1\}$ a $\omega$ je $n$-ta komplexni odmocnina s 1.
Polynom rozdelime na sude a liche mocniny: $P(x)=S(x^2)+xL(x^2)$. Diky tomu
umime rychle vyhodnotit $P(-x)$ jako $S(x^2)-xL(x^2)$.
Vyjde nam to v $\O(N\log N)$.

Reprezentace pres vysledky v bodech je totiz uzitecna na nasobeni polynomu.

Inverzni DFT dostaneme tim, ze na DFT se podivame jako linearni zobrazeni
$f(x)_j=\Omega x=\sum_{k=0}^{n-1} x_k \omega^{kj}$, a $\Omega$ je invertibilni
a $\Omega^{-1}=\frac{1}{n} \bar{\Omega}$, takze staci podelit $n$-kem a konvexne
sdruzit.
\end{document}
